\chapter{The CHERI ABIs}
\label{chp:abis}

The CHERI compiler supports two ABIs, an extended version of the MIPS n64 ABI and the ``sandbox'' ABI where every pointer is a capability.
The sandbox ABI is intended to evolve to a point where \creg{0} is never used.
Currently, \creg{0} is used for globals, but not for any other data accesses.

\section{Register usage}

The compiler's use of capability registers is summarized in \autoref{tbl:reguse}.
Registers \creg{16}--\creg{24} are preserved across calls, as is \creg{0}, which is never modified by the compiler.
In the sandbox ABI, so is \creg{11} (the stack capability).

\begin{table}
	\begin{center}
		\begin{tabu}{ll}
			\toprule
			\headerrow
			Register             & Compiler usage \\
			\midrule
			\reg{v0}             & Contains the method number for cross-domain calls.\\
			\creg{0}             & Used implicitly for all non-capability memory accesses.\\
			\creg{1}--\creg{16}  & Used for arguments in the ``fast'' calling convention. \\
			\creg{1}--\creg{2}   & Code and data capability arguments with the ``ccall'' calling convention. \\
			\creg{3}             & Capability return value. \\
			\creg{3}--\creg{10}  & Capability arguments (caller-save). \\
			\creg{11}            & Stack capability (sandbox ABI). \\
			\creg{12}            & Used with \asm{cjalr} as the destination register (sandbox ABI). \\
			\creg{11}--\creg{15} & Temporary (caller-save) registers.\\
			\creg{17}            & Capability link register used with \asm{cjalr} (sandbox ABI).\\
			\creg{16}--\creg{24} & Saved (callee-save) registers. \\
			\creg{25}--\creg{31} & Not used by the compiler.\\
			\bottomrule
		\end{tabu}
		\caption{\label{tbl:reguse}Capability register usage.}
	\end{center}
\end{table}

\subsection{The sandbox ABI}

The sandbox ABI uses \creg{11} as a stack capability.
The \reg{sp} and \reg{fp} registers contain offsets in the stack capability.
It would make sense to replace \reg{sp} with the offset in \creg{11}.
This would simplify the addressing modes for the stack and make a lot of spills cheaper in the sandbox ABI.

\dcnote{Currently only true for the capcall branch, but will be merged Real Soon Now™}
The sandbox ABI uses the capability mechanism to protect the return address.
In the MIPS ABI, function calls are implemented as \asm{jalr $t9, $ra}.
In the sandbox ABI, they are \asm{cjalr $c12, $c17}.
This has two effects.
The first is that \reg{t9} can no longer be used as a cheap way of getting the program counter for position-independent code.
Instead, if this value is needed, the compiler will emit \asm{CGetOffset $t9,$c12}, setting \reg{t9} to the \reg{pcc}-relative address of the program counter on function entry.

The second effect is that the return can be emitted as \asm{cjr $c17}.%stopzone
If \creg{17} is spilled to the stack, then it is protected by its tag bit.
If it is overwritten by anything other than an executable capability, then the \asm{cjr} instruction will trap.

\section{Calling conventions}

When targeting the n64 ABI, the only changes to the calling convention are to support capability arguments.
Capability arguments are passed in registers \creg{3} to \creg{10}, with \creg{3} also being used for capability return values.

The normal rules for composite types apply: the portions that will fit within a register are passed in registers and the remainder is passed on the stack.

\subsection{Variadic calls}

The n64 ABI requires that all arguments to variadic functions are passed either on the stack or in 8 integer registers.
When a \ccode{va_list} is constructed, the 8 integer values are written out.
This means that capability arguments in variadic functions are very difficult to support and are currently not expected to work.
To fix them, we will have to ensure that all arguments after the first capability argument are passed on the stack and that the first capability argument has a mechanism for knowing whether it is in the first 64 bytes of the \ccode{va_list} (and, if so, incrementing the pointer until it's at the end).

\index{sandbox ABI!variadic functions}
In the sandbox ABI, \textit{all} variadic arguments are passed on the stack, with their natural alignment (non-variadic arguments to variadic functions are passed in registers or on the stack as the normal calling convention would dictate).
The \ccode{va_list} is simply a pointer to the first (on-stack) variadic argument and \ccode{va_arg} calls simply ensure correct alignment, load from the pointer, and increment it past the value.
The alignment requirements can result in large gaps in the variadic argument list.


\section{Cross-domain calls: ``ccall''}

The \ir{chericcallcc} calling convention uses registers \creg{1} and \creg{2} for the first two capability arguments and \reg{v0} for the method number.
The front end will lower \ccode{struct}s that fit in registers to a sequence of scalars, so this is typically generated from a two-capability \ccode{struct}.
The remaining arguments are in the same place as the normal calling convention, allowing a simple jump to tail call functions that do not care about these arguments.

The back end tracks the argument registers that are used by callers of functions with this convention and the return registers that are used by the callee.
At each call site, it will zero unused argument registers.
In the callee, it will zero unused return registers.

\begin{notebox}{Soft float}
	The compiler currently assumes that the soft-float ABI is always used for calls into sandboxes and so does not zero any floating point registers.
	The CHERIBSD trampoline code in the kernel also makes this assumption, but it will need revisiting once floating point is enabled.
\end{notebox}

The code that will be generated is described in \autoref{sec:cccall}.
The compiler assumes that the runtime environment (library or kernel code) is responsible for preserving or zeroing all non-argument registers across security domain transitions.
The code to do this is the same for all functions and so having only a single instance provides better cache utilization.

