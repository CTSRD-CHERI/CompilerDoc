\chapter{The CHERI ABIs}
\label{chp:abis}
\label{chap:abis} % Label in the style of CHERI Programmer's Guide

The CHERI compiler supports two ABIs, an extended version of the \mipsABI{} and the \purecapABI{} where every pointer is a capability.
The \purecapABI{} is intended to evolve to a point where \creg{0} is never used.
Currently, \creg{0} is used for globals, but not for any other data accesses.

\section{Register usage}

The compiler's use of capability registers is summarized in \autoref{tbl:hybridreguse} and \autoref{tbl:purecapreguse}.
The \hybridABI{} is intended as a superset of the \mipsABI{} and so only differs from the \mipsABI{} in support for capabilities.
The \purecapABI{} is a new ABI that is similar to the \mipsABI{} but makes full use of capabilities and so is not intended to be compatible with the \mipsABI{}.

\begin{table}
	\begin{center}
		\begin{tabu}{lccX}
			\toprule
			\headerrow
			Register             & ABI Name            & Saver  & Compiler usage \\
			\midrule
			\reg{0}              & \reg{zero}          & Caller & Constant zero register.\\
			\reg{1}              & \reg{at}            & Caller & Assembler temporary.\\
			\reg{2}              & \reg{v0}            & Caller & First integer return value, method number for cross-domain calls.\\
			\reg{3}              & \reg{v1}            & Caller & Second integer return value.\\
			\reg{4}              & \reg{a0}            & Caller & First integer argument, exception selector.\\
			\reg{5}              & \reg{a1}            & Caller & Second integer argument, exception object.\\
			\reg{6}--\reg{11}    & \reg{a2}--\reg{a7}  & Caller & Integer arguments.\\
			\reg{12}--\reg{15}   & \reg{t4}--\reg{t7}  & Caller & Temporaries.\\
			\reg{16}--\reg{22}   & \reg{s0}--\reg{s6}  & Callee & Saved registers.\\
			\reg{23}             & \reg{s7}            & Caller & Base pointer.\\
			\reg{24}             & \reg{t8}            & Caller & Temporary.\\
			\reg{25}             & \reg{t9}            & Caller & \asm{jalr} target register.\\
			\reg{26}--\reg{27}   & \reg{k0}--\reg{k1}  & N/A    & Reserved for kernel.\\
			\reg{28}             & \reg{gp}            & Callee & Global pointer.\\
			\reg{29}             & \reg{sp}            & Callee & Stack pointer.\\
			\reg{31}             & \reg{fp}            & Callee & Frame pointer.\\
			\reg{31}             & \reg{ra}            & Caller & Return address.\\
			\midrule
			\creg{0}             & \creg{null}         & Callee & Constant NULL capability register.\\
			\creg{1}--\creg{2}   &                     & Caller & Code and data capability arguments for cross-domain calls.\\
			\creg{3}             &                     & Caller & Capability return value.\\
			\creg{3}--\creg{10}  &                     & Caller & Capability arguments.\\
			\creg{11}--\creg{15} &                     & Caller & Temporary registers.\\
			\creg{16}--\creg{25} &                     & Callee & Saved registers.\\
			\creg{24}--\creg{31} &                     & Caller & Not used by the compiler.\\
			\bottomrule
		\end{tabu}
		\caption{\label{tbl:hybridreguse}Interface register conventions (\hybridABI).}
	\end{center}
\end{table}

\begin{table}
	\begin{center}
		\begin{tabu}{lccX}
			\toprule
			\headerrow
			Register             & ABI Name            & Saver  & Compiler usage \\
			\midrule
			\reg{0}              & \reg{zero}          & Caller & Constant zero register.\\
			\reg{1}              & \reg{at}            & Caller & Assembler temporary.\\
			\reg{2}              & \reg{v0}            & Caller & First integer return value, method number for cross-domain calls.\\
			\reg{3}              & \reg{v1}            & Caller & Second integer return value.\\
			\reg{4}              & \reg{a0}            & Caller & First integer argument, exception selector.\\
			\reg{5}--\reg{11}    & \reg{a1}--\reg{a7}  & Caller & Integer arguments.\\
			\reg{12}--\reg{15}   & \reg{t4}--\reg{t7}  & Caller & Temporaries.\\
			\reg{16}--\reg{23}   & \reg{s0}--\reg{s7}  & Callee & Saved registers.\\
			\reg{24}--\reg{25}   & \reg{t8}--\reg{t9}  & Caller & Temporaries.\\
			\reg{26}--\reg{27}   & \reg{k0}--\reg{k1}  & N/A    & Reserved for kernel.\\
			\reg{28}--\reg{30}   &                     & Callee & Saved.\\
			\reg{31}             &                     & Caller & Temporary.\\
			\midrule
			\creg{0}             & \creg{null}         & Callee & Constant NULL capability register.\\
			\creg{1}--\creg{2}   &                     & Caller & Code and data capability arguments for cross-domain calls.\\
			\creg{3}             &                     & Caller & Capability return value.\\
			\creg{3}--\creg{10}  &                     & Caller & Capability arguments.\\
			\creg{11}            & \creg{sp}           & Caller & Stack capability.\\
			\creg{12}            &                     & Caller & \asm{cjalr} target register.\\
			\creg{13}            &                     & Caller & Capability to on-stack arguments.\\
			\creg{11}--\creg{15} &                     & Caller & Temporary registers.\\
			\creg{16}            &                     & Caller & Exception pointer register.\\
			\creg{17}            & \creg{ra}           & Callee & Capability return register.\\
			\creg{24}            & \creg{fp}           & Callee & Capability frame pointer.\\
			\creg{25}            & \creg{bp}           & Callee & Capability base pointer.\\
			\creg{17}--\creg{25} &                     & Callee & Saved registers.\\
			\creg{26}            & \creg{gp}           & N/A & Capability globals pointer (\reg{idc} when used with \asm{CCall}).\\
			\creg{27}--\creg{31} &                     & Caller & Not used by the compiler.\\
			\bottomrule
		\end{tabu}
		\caption{\label{tbl:purecapreguse}Capability register usage. (\purecapABI)}
	\end{center}
\end{table}



Note that the calling convention described here applies only to functions that are part of a module's public ABI.
The compiler is free to use other calling conventions for internal functions.
For example, the current version of LLVM will use all of the (integer and
capability) temporary registers for arguments for internal functions.

\subsection{The \purecapABI{}}

The \purecapABI{} uses \creg{11} as a stack capability (the assembler will also accept \creg{sp} as a name for this register).
Currently, the bounds of this capability do not change during program execution and call frames just adjust the offset.
Because functions are expected to save and restore the stack capability, it is possible to refine this without changing the ABI and in the future the compiler is likely to spill the incoming \creg{sp} and then restrict the bounds of the value in the register to prevent accidental overwrites.

On-stack arguments are accessed relative to \creg{13}.
If there are no on-stack arguments, callers are responsible for ensuring that \creg{13} contains a zero capability.
The compiler currently assumes that a function without on-stack arguments has a NULL \creg{13} on entry so will not attempt to clear it
before calling another function.

%\begin{notebox}{\creg{13} optimisations}
%	Being able to detect if there are on-stack arguments is generally useful for various trampoline functions.
%	In particular, code forwarding to cross-domain calls can copy arguments from the caller's stack to the callee's stack.
%	The compiler currently zeroes this register for every call that doesn't take on-stack arguments (i.e. most function calls).
%	We could most likely avoid a lot of this overhead by simply forwarding \creg{13} and perhaps by making it a callee-saved register so that the null value is preserved across calls.
%\end{notebox}


The following integer registers are special in the \mipsABI{} but have no special meaning in the \purecapABI{}:
\reg{sp}, \reg{fp}, \reg{s7}, \reg{t9}, \reg{ra}.
Of these, \reg{sp} is the only register that is not already marked as either caller- or callee-saved by the \mipsABI{} and becomes a temporary (caller-save) register in the \purecapABI{}.

The \purecapABI{} uses the capability mechanism to protect the return address.
In the MIPS ABI, function calls are implemented as \asm{jalr $t9, $ra}.
In the \purecapABI{}, they are \asm{cjalr $c12, $c17}.
This has two effects.
The first is that \reg{t9} can no longer be used as a cheap way of getting the program counter for position-independent code.
Instead, if this value is needed, the compiler will emit \asm{CGetOffset $t9,$c12}, setting \reg{t9} to the \reg{pcc}-relative address of the program counter on function entry.

The second effect is that the return can be emitted as \asm{cjr $c17}.%stopzone
If \creg{17} is spilled to the stack, then it is protected by its tag bit.
If it is overwritten by anything other than an executable capability, then the \asm{cjr} instruction will trap.

\section{Calling conventions}

When targeting the \mipsABI{}, the only changes to the calling convention are to support capability arguments.
Capability arguments are passed in registers \creg{3} to \creg{10}, with \creg{3} also being used for capability return values.

The normal rules for composite types apply: the portions that will fit within a register are passed in registers and the remainder is passed on the stack.

\subsection{Variadic calls}

The \mipsABI{} requires that all arguments to variadic functions are passed either on the stack or in 8 integer registers.
When a \ccode{va_list} is constructed, the 8 integer values are written out.
This means that capability arguments in variadic functions are very difficult to support and are currently not expected to work.
To fix them, we will have to ensure that all arguments after the first capability argument are passed on the stack and that the first capability argument has a mechanism for knowing whether it is in the first 64 bytes of the \ccode{va_list} (and, if so, incrementing the pointer until it's at the end).

\index{\purecapABI{}!variadic functions}
In the \purecapABI{}, \textit{all} variadic arguments are passed on the stack, with their natural alignment (non-variadic arguments to variadic functions are passed in registers or on the stack as the normal calling convention would dictate).

The \creg{13} register holds a capability to the on-stack arguments.
The \ccode{va_start} function copies the value that was stored in \creg{13} on entry to the function.

The \ccode{va_list} is a capability to the (on-stack) variadic arguments and \ccode{va_arg} calls ensure correct alignment, load from the capability, and increment its offset past the value.
The alignment requirements can result in large gaps in the variadic argument list if integer and capability arguments are interleaved.

\section{Cross-domain calls: ``ccall''}

The \ir{chericcallcc} calling convention uses registers \creg{1} and \creg{2} for the first two capability arguments and \reg{v0} for the method number.
The front end will lower \ccode{struct}s that fit in registers to a sequence of scalars, so this is typically generated from a two-capability \ccode{struct}.
The remaining arguments are in the same place as the normal calling convention, allowing a simple jump to tail call functions that do not care about these arguments.

The back end tracks the argument registers that are used by callers of functions with this convention and the return registers that are used by the callee.
At each call site, it will zero unused argument registers.
In the callee, it will zero unused return registers.

\begin{notebox}{Soft float}
	The compiler currently assumes that the soft-float ABI is always used for calls into sandboxes and so does not zero any floating point registers.
	The CHERIBSD trampoline code in the kernel also makes this assumption, but it will need revisiting once floating point is enabled.
\end{notebox}

The attributes for generating this code are described in \autoref{sec:cccall}.
The compiler assumes that the runtime environment (library or kernel code) is responsible for preserving or zeroing all non-argument registers across security domain transitions.
The code to do this is the same for all functions and so having only a single instance provides better cache utilization.

The compiler generates two special sections containing metadata about cross-domain calls.
\index{\_\_cheri\_sandbox\_required\_methods section}
\asm{__cheri_sandbox_required_methods} contains metadata about methods that are required by the binary.
\index{\_\_cheri\_sandbox\_provided\_methods}
\asm{__cheri_sandbox_provided_methods} contains metadata about methods that are provided by this binary.
These are used by the \file{libcheri}\index{libcheri} runtime, to make method numbers in the caller and callee of cross-domain calls match.

For each method that is defined in the binary, the \asm{__cheri_sandbox_provided_methods} section will contain an instance of the following structure:

\begin{csnippet}
struct sandbox_provided_method
{
	int64_t   flags;
	char     *class;
	char     *method;
	void     *method_ptr;
};
\end{csnippet}

The \ccode{flags} field is currently always 0 and is reserved to permit future modifications to this structure without breaking compatibility.
The \ccode{class} field points to the name of the class for which the method is provided.
The \ccode{method} field points to the name of the method.
The \ccode{method_ptr} field is a \reg{pcc}-relative address of the method.

Similarly, for each required method the compiler will emit a structure of the following form in the \asm{__cheri_sandbox_required_methods} section:

\begin{csnippet}
struct sandbox_required_method
{
	int64_t   flags;
	char     *class;
	char     *method;
	int64_t  *method_number_var;
	int64_t   method_number;
};
\end{csnippet}

Multiple compilation units in the same binary may require the same method, so each of these structures is emitted as a single \keyword{comdat}, to allow merging.
Older versions of the cross-domain call ABI put the method numbers in a separate section and required the runtime library to walk the ELF symbol table.
The current structure is a hybrid, with the \ccode{method_number_var} field containing the address of that global, allowing code to be compiled and used with old and new versions of the runtime library.

The \ccode{flags} field is used by the runtime to indicate that a method has been resolved.
It will also be used in a future version to indicate that the \ccode{method_number_var} field has been omitted and that the generated code expects the \ccode{method_number} field to contain the authoritative version of the method number.
The top 32 bits of the \ccode{flags} field are reserved for use by the runtime, the low 32 bits for use by the compiler.

The \ccode{class} and \ccode{method} fields contain the names of the class and method, respectively.

\subsection{CHERI Callbacks}

Function pointers annotated with the \ccode{cheri_callback} attribute are represented as a structure of the following form:

\begin{csnippet}
struct __cheri_callback
{
	__capability void *code;
	__capability void *data;
	uint64_t           mathod_number;
}
\end{csnippet}

\section{Global initialization}
\index{global initializers}
Capabilities in globals require special handling.
Capabilities can not be statically defined in the binary as other data, because doing so will not set the tag.
Similarly, existing relocation types are not sufficient to describe a capability, which has a base, bounds, and permissions in addition to the location described by conventional pointers.

\begin{notebox}{Dynamic initialization}
The initial implementation in clang simply emitted C++-style dynamic initialization code for all globals that contained non-null pointer values in the constant initializer.
This is no longer the default, but it can be enabled with the \flag{-Xclang -no-cheri-linker} flag.
\end{notebox}
\subsection{In statically linked binaries}

The LLVM back end will emit a special section in the ELF binary for these initializers.
This section contains one entry for each capability that must be initialized at program launch.
For example, consider the following program fragment:
\begin{csnippet}
extern int a[5];
int *b[] = {&a[2], &a[1], a};
\end{csnippet}

The resulting binary will contain a \asm{__cap_relocs}\index{\_\_cap\_relocs section} section with three instances of the following structure:

\begin{csnippet}
struct capreloc
{
	void *__capability capability_location;
	void              *object;
	uint64_t           offset;
	uint64_t           size;
	uint64_t           permissions;
};
\end{csnippet}

The \ccode{capability_location} field contains the (relative) address of the capability that must be initialised at run time.
The \ccode{object} field contains an address (and associated relocations) of the object that the capability refers to.
The \ccode{offset} field contains the offset within this object.
The \ccode{size} field contains the size of the underlying object.
The \ccode{permissions} field contains the permissions that the capability should have and reserves space for other flags.

For the above example, the compiler will emit three structures, with the following values:

\begin{csnippet}
	{ &b[0], &a, 8, 0, 0},
	{ &b[1], &a, 4, 0, 0},
	{ &b[2], &a, 0, 0, 0}
\end{csnippet}

The compiler does not know the size of the object and so will set the size to 0.
After linking, the ELF file will contain the size of the symbol.
A capability-aware linker would then fill in the size field.
In the absence of such a linker, the \tool{capsizefix} tool will set the size.

Currently, the permissions field has the high bit set for functions and is otherwise zero.

You can check wither this has worked by using the \flag{-C} flag to \tool{llvm-objdump}:

\begin{verbatim}
$ llvm-objdump -C a.out

a.out:	file format ELF64-mips

CAPABILITY RELOCATION RECORDS:
0x000000000000a7a0	Base: a (0x000000000000a700)	Offset: 0000000000000008	Length: 0000000000000020
0x000000000000a7c0	Base: a (0x000000000000a700)	Offset: 0000000000000004	Length: 0000000000000020
0x000000000000a7e0	Base: a (0x000000000000a700)	Offset: 0000000000000000	Length: 0000000000000020
\end{verbatim}

There are currently some significant limitations:

\begin{itemize}
	\item We have no way of enforcing permissions (for example, the \ccode{__output} or \ccode{__input} qualifier on pointers).

\end{itemize}

\subsection{In dynamically linked binaries}
In dynamically linked binaries the \asm{__cap_relocs} section is only used in order to initialize local (non-premptible) symbols.
For references to symbols in other shared objects the static linker (LLD) will emit a \asm{R_MIPS_CHERI_CAPABILITY} relocation.
In order to resolve this relocation, the dynamic linker should use the address of the target symbol as the base for the resulting capability.
Furthermore, the relocation addend should be used as the offset of the resulting capability.
There is currently no mechanism to set permissions on these capabilities. However, for \asm{STT_FUNC} type symbols the capability must be
derived from an executable capability, and for all other symbol types the resulting capability should not be executable.

\begin{notebox}{Capability length }
	The length should be taken from the \asm{Elf_Sym} \asm{st_size} field unless the symbol type is \asm{STT_FUNC}.
	However, for \asm{STT_FUNC} type symbols it should set the size tothe size of the containing shared objects
	text+read-only data segment so that global read-only data can be derived from it.
	This restriction is necessary because the compiler currently derives the capability for the global capability table
	(which is used for every C-language global variable access and function call) from the current \reg{pcc}.
	There are experimental compiler modes that would allow setting tight bounds on functions (such as \flag{-cheri-cap-table-abi=plt})
	but they still have known limitations and should not be used yet.
\end{notebox}

\paragraph{Legacy mode}
The \asm{__cap_relocs} mechanism for external symbols can still be used for dynamically linked binaries. This may be useful if it is not possible to change the dynamic linker. However, it is rather fragile and inefficient:
\begin{itemize}
	\item The \asm{capability_location} field needs a dynamic relocation to add the load address. This is needed so that the initialization code can determine the correct write location.
	\item The \asm{object} field needs a dynamic relocation which resolves to the external symbol. For local symbols this is a relative relocation that adds the load address of the shared library.
	\item For correctness every entry in the array also needs a \asm{R_MIPS_CHERI_SIZE} dynamic relocation for the size. Originally, the static linker would just write the \asm{st_size} value of the symbol at static link time (or 0 if not found) to the size field. However, for weak symbols and symbols that might be resolved from a different shared library at runtime this value might not be correct.
\end{itemize}

This old method can still be enabled by passing \flag{-Wl,--preemptible-caprelocs=legacy} but should not be used if the dynamic linker can be updated to handle the \asm{R_MIPS_CHERI_CAPABILITY} relocation.


\section{Return address protection}

RISC architectures typically provide a jump instruction that puts the return address in another register (e.g. \asm{jalr} on MIPS, \asm{bl[x]} on ARM).
If the called function calls another function, it must spill the return address to the stack, where it can be reloaded later.
This happens automatically on x86, where the \asm{call} and \asm{ret} instructions store the return address on the stack and read it from the stack, respectively.

If a buffer overflow allows the return address to be overwritten, then an attacker can control exactly where execution will continue after the return.

\index{CFI}\index{return address}
In the \purecapABI{}, this kind of attack is very difficult.
Calls use the \asm{cjalr} instruction, so the return address is a \reg{pcc}-relative capability.
If this is overwritten with something that is not a capability, then the return will trigger a tag violation.
If this is overwritten by a non-executable capability, then the return will trigger a permissions violation.
For a successful exploit, the attacker would have to find an executable capability (e.g. a function pointer or a previous return address) that the program could be tricked into writing over the return address.

We would like to be able to provide the same benefits to the existing ABI.
We achieve this by keeping the call sequence the same, but modifying the return sequence.
In all non-leaf functions, when we spill the return address to the stack we also spill a \keyword{return capability}: \reg{pcc} with its offset set to the return address.
This can then be used with \asm{cjr} to return.
Note that we still spill the return address, even though it is not used, because other tools (debugging tools and so on) sometimes rely on the position of the return address on the stack.


\section{C++ Exception handling}

Exception handling in CHERI follows the Itanium C++ ABI specification~\cite{noauthor_ABIItaniumException_2018}, with some minor changes.
In this model, each function has a \asm{.eh_frame} entry associated with it, describing to a generic unwind table the location of all saved registers and how to unwind the stack and restore all register values.
The \asm{.eh_frame} entry also points to a \keyword{personality function} and a \keyword{language-specific data area} (LSDA)~\cite{Hewlett-Packard_ExceptionHandlingTables_1999}.
The personality function is responsible for parsing the LSDA and informing the unwinder if, for the call site where an exception was thrown, the frame needs to either catch the exception (stop unwinding) or run some cleanup code.
In both cases, the personality function will update the saved registers to transfer control to the landing pad that corresponds to the call site and inform it of the relevant action.

In the \purecapABI{}, the pointer to the personality function is a capability.
Moreover, the individual call site records no longer store offsets relative to \texttt{@LPStart} (which is generally the start of the function)~\cite[\S{}7.3.2]{Hewlett-Packard_ExceptionHandlingTables_1999}, but instead stores the handler as an additional capability a capability.\footnote{%
  Changing these offsets into capabilities is not ideal from a start-up performance point-of-view since it requires one additional dynamic relocation to be processed for every \cxxcode{catch} statement.
  However, this change is required in order to support immutable (\textit{sentry}) code pointers.}
To allow using a single byte for \ccode{NULL} landing pads (instead of requiring a capability-aligned and capability size zero value), the landing pad is still stored as a ULEB128 value.
However, if the the landing pad has the special value \ccode{0xc}, the real landing pad is stored as a capability at the next capability-aligned address (and the \textit{action record} field follows this capability).\footnote{%
  See \url{https://github.com/CTSRD-CHERI/libcxxrt/commit/f8e49399} for the \file{libcxxrt} implementation of this ABI difference.}

% LLVM changes: \url{https://github.com/CTSRD-CHERI/llvm-project/commit/20df223ea42b11a43da3b8ffc3601c631a8ffda2} and \url{https://github.com/CTSRD-CHERI/llvm-project/commit/23f555c23761b0299ba61abc137016e63e7c4120}
% libcxxrt changes: \url{https://github.com/CTSRD-CHERI/libcxxrt/commit/f8e4939925f1cb9b8021cc88eed62bff9f4d2311}

\subsection{Register usage}

The Itanium ABI reserves two registers for communicating between the personality function and the landing pad.
These are referred to as the exception pointer register and the exception selector register.
The first contains the address of the thrown exception, the second the index in the type table for the exception, used by the landing pad code to branch to the correct catch or cleanup block.
For existing targets, both of these are general purpose registers.

In the \purecapABI{}, the exception register (\creg{16}) is a capability register, granting access to the exception.
The exception selector register remains an integer and we use the same register as the legacy ABI for this purpose.

\subsection{Unwind tables}

The unwind table format is unmodified.
The tables contain a large number of addresses, but these are not transformed into capabilities.

In existing implementations, the exception metadata is often as big as the code and increasing the size is unacceptable.
ARM and Apple have compact versions of the unwind tables, which increase the density further.
In particular, addresses are often encoded as offsets and as ULEB128 values (variable-length encodings, optimised for small values) and are stored with no padding.

The size is not important in most uses, because the unwind tables are not paged in from persistent storage unless an exception is thrown.
Transforming the addresses into capabilities would require that the run-time linker or equivalent initialise them with valid tagged capabilities on startup, which would impose a considerable start-up time penalty.

Instead, all addresses in the uwnind tables are treated as being relative to an existing capability.
In the current prototype, they are treated as relative to \reg{pcc}, but in the future most should be derived from a read-only (non executable) capability to the mapped binary or from the current frame's stored link register.
The only exception would be the address of the personality function, though this could be addressed by adding a layer of indirection.
ARM's 32-bit exception ABI (ARM EHABI) follows a similar route, replacing the address of the personality function with an index into a table of personality functions (a significant space saving, given that most programs have at most three distinct personality functions).
Requiring the run-time linker to resolve the capabilities for these functions in a table would add negligible overhead.

\subsection{Name Mangling}

Name mangling in CHERI follows the Itanium C++ ABI specification with the addition that \ccode{__capability}-qualified pointer and reference types are prefixed with the extended qualifier \ccode{U3cap}.
This allows function overloading in C++ when parameter types only differ in whether they are \ccode{__capability}-qualified or not.
Here are some examples of functions with capability parameters and their corresponding mangled names:

\begin{csnippet}
void f(int * __capability); // _Z1fU3capPi
void g(int & __capability); // _Z1gU3capRi
void h(int && __capability); // _Z1hU3capOi
\end{csnippet}
