\chapter{Abstract model}
\label{chap:abstract-model}

To understand CHERI as a programmer target, it is important to understand the abstract machine that CHERI exposes.  This is intended to be compliant with C abstract machine, yet provide strong memory safety guarantees that can serve as foundations for security properties.

\section{Capabilities as pointers}

CHERI provides \keyword{memory capabilities} to limit access to \keyword{virtual memory}.  Software can only access the subset of its virtual address space for which it has valid capabilities.  Each load or store operation---including instructions and instruction fetch---rely on implicit or capabilities.  MIPS load and store operations are indirected via the \keyword{ambient data capability} (\creg{0}), CHERI load and store operations take explicit capability operands.

Capabilities grant some permissions on a range of memory, identified by a base and a length.  The CHERIv3 ISA extends this to incorporate an \keyword{offset}, intended to make it possible to use capabilities, rather than integers, as pointers in C-like languages.  You can perform arbitrary arithmetic on a capability's offset, but can only dereference it if it lies within the bounds and you have the relevant permissions.

\section{Operations on capabilities}

CHERI does not allow arbitrary privilege elevation.  A piece of code can disclaim a capability, by simply overwriting it with some data.  It can also limit a capability by reducing its length, increasing its base (which decreases its length by a corresponding amount), or reducing the set of permissions.  All of these operations provide a monotonic decrease in the access that a capability grants.  If you have a capability to an allocation, for example an array, then you can construct capabilities to a range within the array, or capabilities that can only read or write (but not both) to the array, but you can not use it to construct a capability that allows you to write past the end of the array.

The offset, as previously mentioned, is exempt from the monotonic property.  It may be set to any value, but any load and store that attempts to access outside of the permitted range will fail.  

\section{Integers in capabilities}

One important property of C is the existence of types such as \ccode{intptr_t}, that can store either a pointer or an integer and can be operated on as if they were integers.  Integers can be stored in capabilities using the offset: a null capability grants no rights, but can have an offset anywhere in the virtual address space.  The offset can therefore be used to store integer values.

\index{intmax\_t type}
The current implementation defines \ccode{intmax_t} to be \ccode{long}.
This is allowed by the C specification, as the requirement for \ccode{intmax_t} is that it must have a \textit{range} equal to the largest integer, not a \textit{size}.
We believe that this is likely to cause fewer problems in porting code, as most uses of \ccode{intmax_t} are for storing integer, not pointer, values.


\section{The different ABIs}

Incremental adoption is one of the goals of CHERI.  All existing unmodified binaries for the base architecture are expected to work.  More importantly, it should be possible to run unmodified programs with modified libraries and vice versa.

With this in mind, we support two ABIs, discussed more in \autoref{chp:abis}.  One is a small extension to the base ABI, permitting \textit{some} pointers to be capabilities, but with the expectation that \creg{0} encompasses the entire (or majority of the) address space and that the memory protection can be deliberately bypassed.  The second, the \purecapABI{}, is intended to be used when compatibility is not a concern and uses capabilities for all pointer, as well as for the stack.

With care, it is possible to mix the use of both ABIs within a single program.  The layout of various data structures will be different, so the programmer is responsible for ensuring that data is correctly marshaled at boundaries.

\section{Protection properties}

CHERI C model aims to refine the C abstract machine in such a way that undefined and implementation-defined behaviour in C become well-defined behaviour in CHERI-C.
In the \purecapABI{}, many of these properties are guaranteed, whereas in the \hybridABI{} they are best-effort.

\subsection{Object bounds}

Objects in C are either in automatic storage (stack allocations), allocated (malloc and similar), static (globals and statics), or thread (\ccode{_Thread_local} globals and statics).
All accesses to any of these should be via a capability in the \purecapABI{}.
Of these, allocated objects are only ever referred to by a pointer, the others can be referred to by name and can have the address of that name taken.
In an implementation with precise capabilities, for example CHERI256), the capability representing the pointer to any object created in this way must have the exact bounds of the object.
In an implementation with imprecise capabilities, for example CHERI128, the bounds of the capability may be larger than the size of the type but must not grant access to any other objects.

In the \hybridABI{}, these are best effort.
It is recommended that at least the following create capabilities with the same bounds as their equivalents in the \purecapABI{}:
\begin{itemize}
	\item Directly casting the result of a \ccode{malloc}, \ccode{calloc}, \ccode{valloc}, or \ccode{posix_memalign} call to a \ccode{__capability}-qualified pointer.
	\item Directly casting the address of a local variable with automatic storage to a \ccode{__capability}-qualified pointer.
	\item Directly casting the address of a \ccode{static} or global variable to a \ccode{__capability}-qualified pointer.
\end{itemize}

Compilers are free to transform other pointers into capabilities if the are guaranteed not to cross an ABI boundary.
This should have no observable impact unless the code exhibits undefined behaviour by accessing beyond the bounds of an object and simply constrains the undefined behaviour to raising a signal.

\subsection{Function pointer protection}

Capabilities to normal allocations should not have the executable permission set.
Function pointers in the \purecapABI{} should have the executable permission but no store permissions set.
This ensures that pointers to code and to data cannot be confused.
There is no mechanism for providing this separation in the \hybridABI{} for non-annotated pointers.

\subsection{Return address protection}

In the \purecapABI{}, the return address should always be a(n executable) capability.
If it is spilled to the stack, then the tag bit and the execute permission will protect it from being overwritten by anything other than a function pointer.
In the \hybridABI{}, it is recommended that spills of the return address to the stack instead spill a \reg{pcc}-derived capability, affording the same protection.
This can be done in an ABI-preserving way by spilling the return capability in addition to the return address, allowing simple stack walkers to continue to function.

\subsection{C++ protection}

In C++, as in C, all objects should be referenced via a capability.
Both pointers and references in the \purecapABI{}, and qualified pointers and references in the \hybridABI{} are represented as capabilities.

Unfortunately, in C++ it is common to store objects within arrays (in C, arrays of structs are regarded as being a single object).
The current CHERI ABI does not provide protection in this case, though we have considered several approaches to strengthening this, for example by making arrays containing objects read only capabilities and requiring code that accesses them to load a capability to each object from the object header.
Variations on this could prevent counterfeit object oriented programming (COOP) attacks, were overlapping objects are created by the attacker.

There are several places where the sealing mechanism could enhance protection for C++.
Most notably, vtable pointers and \ccode{type_info} object pointers are very important for C++ type safety and so could be represented as sealed capabilities.
This would prevent user code from creating fake vtables and would make creating the overlapping objects required for COOP-style attacks difficult unless the attacker was already able to read from arbitrary memory addresses and copy the contents to the compromised buffer.

Simply making the vtable pointers unsealed capabilities (as we have done) prevents some attacks of this nature, because the attacker cannot create vtable pointers by simply providing data over an IPC channel.
